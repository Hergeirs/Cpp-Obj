\documentclass[11pt]{article}

%Gummi|065|=)
\title{\textbf{Lab rapport in C++ OOP}}
\author{Hergeir Winther Lognberg \\
Hewi1600}
\date{}
\usepackage{graphicx}
\usepackage{hyperref}
\usepackage{listings}
\usepackage{xcolor}
\usepackage{textcomp}

\definecolor{listinggray}{gray}{0.9}
\definecolor{lbcolor}{rgb}{0.9,0.9,0.9}
\lstset{
backgroundcolor=\color{lbcolor},
	tabsize=4,    
%   rulecolor=,
	language=[GNU]C++,
		basicstyle=\scriptsize,
		upquote=true,
		aboveskip={1.5\baselineskip},
		columns=fixed,
		showstringspaces=false,
		extendedchars=false,
		breaklines=true,
		prebreak = \raisebox{0ex}[0ex][0ex]{\ensuremath{\hookleftarrow}},
		frame=single,
		numbers=left,
		showtabs=false,
		showspaces=false,
		showstringspaces=false,
		identifierstyle=\ttfamily,
		keywordstyle=\color[rgb]{0,0,1},
		commentstyle=\color[rgb]{0.026,0.112,0.095},
		stringstyle=\color[rgb]{0.627,0.126,0.941},
		numberstyle=\color[rgb]{0.205, 0.142, 0.73},
%        \lstdefinestyle{C++}{language=C++,style=numbers}’.
}
\lstset{
	backgroundcolor=\color{lbcolor},
	tabsize=4,
  language=C++,
  captionpos=b,
  tabsize=3,
  frame=lines,
  numbers=left,
  numberstyle=\tiny,
  numbersep=5pt,
  breaklines=true,
  showstringspaces=false,
  basicstyle=\footnotesize,
%  identifierstyle=\color{magenta},
  keywordstyle=\color[rgb]{0,0,1},
  commentstyle=\color{purple},
  stringstyle=\color{red}
  }


\begin{document}

\maketitle

\section{Preamble}

Assignment was to create a list object with the ability to handle and manipulate 20 items of the type int and float. I am supposed to use the standard library functionality everywhere it's practical to do so.
\section{The Code}
\subsection{placement}
I've decided to keep all files associated with the lab in the root of the project folder.
\subsection{Template memberfunction definitions}
Also relevant is it that I have actually defined all template class functions in a cpp file and instantialized with double there. So that the compiler actually would know which template to compile. (hope i'm being clear enough)

\begin{lstlisting}
	//bottom of DataFileReader.cpp
	template class DataFileReader<double>; //instantiating template with type double
\end{lstlisting}

\subsection{namespace}
I chose to remove
\begin{lstlisting}
using namespace std;
\end{lstlisting}
from all files it was previously used in. I find that this improves readability and clearly seperates std functions from self-made ones.


Still I have not had the need (nor want) for a shared pointer. 

\subsection{classes}

In this lab i've constructed 3 classes of significanse to the lab. These are
\begin{itemize}
	\item ListManipulator
	\begin{itemize}
		\item class handles all interaction with list object
	\end{itemize}
	\item TestApp
	\begin{itemize}
		\item class handles all interaction between user and ListManipulator after datatype has been chosen
	\end{itemize}
	\item Interface
	\begin{itemize}
		\item handles running correct version of TestApp. User chooses datatype here.
	\end{itemize}
\end{itemize}

\subsubsection{ListManipulator}
This is a standard template class.

I simply implemented all list aggregations, filling, clearing... etc, here. The results are returned up to Testapp implementations.
Get's instantialized with same datatype as TestApp.

\subsubsection{TestApp}
This is also a template class.

I use this class for giving user results from ListManipulator functions and so on.

\subsubsection{Interface}
Responsible for instantialising (and running) correct version of testapp. This is done using the following function:
\newpage

\begin{lstlisting}
	template <typename T>
	void runMain()
	{
		//run temporary instantialization of TestApp with user defined type 
		TestApp<T>().run(); 
	}	
\end{lstlisting}

and the menu switch looks like this:

\begin{lstlisting}
	void Interface::run()
	{
		//runing interface wrapper.
		bool again=true;
		while(again)
		{
			switch(menu.getMenuChoice())
			{
				case 1:                 //user chooses int
					runMain<int>();
					break;
				case 2:                 //user chooses double
					runMain<double>();	
					break;
				case 3:
					loadFromFile();
					break;
				case 4:
					again=false;
				default:
					break; //won't reach over menuSize anyway
			}
		}
	}\end{lstlisting}

I give user the ability to load before datatype is chosen. In this case the type get's determined by the first char in the "list.dat" file.

\subsection{loading and saving}
	in case of saving to file I always save the datatype in the first line using the following compiler dependant function.
\begin{lstlisting}
	typeid(T).name(); //outputs compiler dependent name for type
\end{lstlisting}

I used this, because as long as I save to file using the same binary as I read from file with. It will work as long as I use the same function to campare with when I read from file.

Therefore the only drawback would be if I try to read a "list.dat" created by a binary compiled with different compiler than my own.

I write the typename to file like this:

\begin{lstlisting}
	ofstream os("list.dat");
	os << typename(T).name() << std::endl;
\end{lstlisting}

and deduce the datatype when loading like this:

\begin{lstlisting}
	// loading from file
	void Interface::loadFromFile()
	{
		std::ifstream is("list.dat");
		char type;
		is >> type;
		if (type==*typeid(int).name()) // check for filtype
		{
			loadRun<int>();     //run apropriate run function.
		}
		else if(type==*typeid(double).name())
		{
			loadRun<double>();
		}
		else
		{
			printPrompt("ERROR TYPE IN FILE","ERROR");
		}
	
	}
\end{lstlisting}
 
and loadRun is implemented like this:

\begin{lstlisting}

	template <typename T>
	void loadRun()
	{	
		// loading list into Testapp and running it
		TestApp<T>().loadFromFile().run();  
	}
\end{lstlisting}

This works as i made the load implementation public in the TestApp class.

\subsubsection{loading after initialisation}
If user tries to load a "list.dat" loaded with wrong datatype a runtime error 
is thrown. Therefore I use a try and catch block within TestApp to handle the 
error and give feedback to user. Nothing is added to list, and app continues to run.

Relevant code:
\begin{lstlisting}
//TestApp.cpp:
template<typename T>
TestApp<T> &TestApp<T>::loadFromFile()
{
	try
	{
		theList->readFromFile();
		menu.enableAll(); //enable all menuoption once list is loaded
		printPrompt("elements from file loaded into list");
	}
	catch (std::runtime_error & error)
	{
		printPrompt("Elements in file are of wrong type","ERROR");
	}
	
	return *this;
}
\end{lstlisting}

\begin{lstlisting}
//ListManipulator.cpp
template<typename T>
void ListManipulator<T>::readFromFile()
{
	std::ifstream is("list.dat");
	if(is)
	{
		clearList();	// in case something already is in list.
		if (is.get()!= *typeid(T).name())
		{
			throw std::runtime_error("wrong type");
		}
		std::istream_iterator<T> eos;
		std::istream_iterator<T> iit(is);
		std::copy(iit,eos,std::back_inserter(*theList));
	}
	is.close();
}
\end{lstlisting}



\section{conclusion}
I believe I found a good solution to the assignment. I could have used polymorphism but chose not to as I believed It wouldn't solve the problem of loading from file as elegantly as the current solution

\section{Building/Compiling}
Just run \emph{make} in the Lab directory.
To run the program run \emph{make run} in same directory.

\section{Enviroment}
I'm programming on an Arch linux 64-bit system. \\ \\
I've got the gcc compiler installed and compile using it's g++ alias which links necessary libraries automatically. \\ \\
To compile I use the recommended flags: "-std=c++11 -Wall -pedantic". The flags let me choose to use c++11 standard and give me useful compiling warnings and errors. 
For editing of code i currently use VS code with a makefile.

\section{Backup}
And if anything's missing you can find it on: \\
github: \url{https://github.com/Hergeirs/Cpp-Obj/tree/master/Level%202/Lab5} \\
\href{https://github.com/Hergeirs/Cpp-Obj/tree/master/Level%202/Lab5}{Cpp-obj/Lab2}

\flushright{\today}
\end{document}
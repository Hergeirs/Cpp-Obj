\documentclass[11pt]{article}

%Gummi|065|=)
\title{\textbf{Project in C++ OOP}}
\author{Hergeir Winther Lognberg \\
Hewi1600}
\date{}
\usepackage{graphicx}
\usepackage{hyperref}
\usepackage{listings}
\usepackage{xcolor}
\usepackage{textcomp}

\definecolor{listinggray}{gray}{0.9}
\definecolor{lbcolor}{rgb}{0.9,0.9,0.9}
\lstset{
backgroundcolor=\color{lbcolor},
    tabsize=4,    
%   rulecolor=,
    language=[GNU]C++,
        basicstyle=\scriptsize,
        upquote=true,
        aboveskip={1.5\baselineskip},
        columns=fixed,
        showstringspaces=false,
        extendedchars=false,
        breaklines=true,
        prebreak = \raisebox{0ex}[0ex][0ex]{\ensuremath{\hookleftarrow}},
        frame=single,
        numbers=left,
        showtabs=false,
        showspaces=false,
        showstringspaces=false,
        identifierstyle=\ttfamily,
        keywordstyle=\color[rgb]{0,0,1},
        commentstyle=\color[rgb]{0.026,0.112,0.095},
        stringstyle=\color[rgb]{0.627,0.126,0.941},
        numberstyle=\color[rgb]{0.205, 0.142, 0.73},
%        \lstdefinestyle{C++}{language=C++,style=numbers}’.
}
\lstset{
    backgroundcolor=\color{lbcolor},
    tabsize=4,
  language=C++,
  captionpos=b,
  tabsize=3,
  frame=lines,
  numbers=left,
  numberstyle=\tiny,
  numbersep=5pt,
  breaklines=true,
  showstringspaces=false,
  basicstyle=\footnotesize,
%  identifierstyle=\color{magenta},
  keywordstyle=\color[rgb]{0,0,1},
  commentstyle=\color{purple},
  stringstyle=\color{red}
  }


\begin{document}

\maketitle

\section{Preamble}

Assignment was to create a dynamic array class which acted like a queue and held the array using a smart pointer.

\section{The Code}
\subsection{placement}
I've decided to keep all files associated with the lab in the root of the project folder. There are in all 12 files 2 for each class and 2 for often used functions. 

\subsection{code}
I chose to implement the class as it was put up to in the Lab. I see that many implemented the array with the type:
\begin{lstlisting}
std::unique_ptr<unique_ptr<int>[10]> arr;
\end{lstlisting}

I didn't see any reason (nor gain) for adding smartpointers for all elements in the array, when we are using a simple standard type, such as int which cleans itself when it goes out of scope. And since the Lab didn't specify this as a requirement I chose simplicity.
\subsection{Question}
Are we still obligated to comply with the c++11 standard? Using:
\begin{lstlisting}
smartPtr = std::make_unique<int[10]>();
\end{lstlisting}

Breaks that requirement as the feature was not introduced until c++14.

Because I've been told to comply with c++11 I've chosen to make the constructor as such.

\begin{lstlisting}
void TestApp::createQueue()
{
	int size;
	do
	{
		getInt(size,"enter queue size (must be more than 1): ");
	} while (size<1); 	//a size under 1 is illegal and doesn't make sense (defensive programming)

	std::unique_ptr <Queue> temp(new Queue(size));
	queue = std::move(temp);
}
\end{lstlisting}

I create a temporary smartpointer containing a pointer to the new dynamically created int array and move It into our private class variable smartpointer queue.
The temporary smartpointer deletes itself as soon as it leaves scope and thereby leaks no memory.

In Queue class I also avoided 'make unique' by using initializer list for getting size of the dynamic int array as follows. 

\begin{lstlisting}
Queue::Queue(size_t size)
:queue(new int[size])
{
	maxElem=size;
	head=0;	 // first element in array.
	tail=-1; // starting with -1 removes a special case in enqueue.
	nElem=0; // nElem keeps track of amount of integers in array.
}
\end{lstlisting}

This upholds the requirements as Queue is not supposed to be able to change size after initialization, but to be initialized with different sizes. Also It's dynamic as it only initializes the array after runtime using the new operator.




I've kept all overloadings of operators at the bottom of the relevant files to keep consistent style.

In \emph{Jukebox.cpp} (the most substantial file) I've tried to make the placement of functions chronological to the menu switch statement orders. And functions not called directly by switches at the bottom of the file. 


The code in the \emph{functions.cpp}. Here i keep some functions I've made myself for handling user input (\emph{getInt,getLine}). And some functions to format prompts and other input in a pleasing way. Also i keep the 
\emph{"const char DELIM"} and at the top a typedef to give \emph{unsigned int} a shorter alias as \emph{uint}

Also i chose to keep my home-brewed \emph{toCase} function instead of using \emph{transform} in cooperation with to-upper/lower because of the gained simplicity of use in our case.

Then there's the main function which contains nothing but the creation of a \emph{Jukebox} object and running it's run function. 

The rest of the files in the project are pairs of class header and definition files.

I really hope my comments will help to explain enough.

\subsubsection{jukeBox}
handles all user interaction and keeps track of it's own albums vector.

\subsubsection{Menu}
\emph{getMenuChoice} this function takes care of everything related with user input and menu presentation. The function only accepts an integer input from user, and even then it checks whether the menuchoice exists and is enabled. If not it re-prompts user. The only non-menuItem entry it accepts is the last switch entry (the exit/return) condition.
\subsubsection{Queue / Playlist}
I have created a Queue that completes all requirements this assignment has set forth. It grows by 5 spaces every time it it's getting too small. And it's using indexes instead of pointers to keep track of first and last element in list. (Personally I would have used real pointers)

As It should be, user has no control over anything except: 
\begin{itemize}
\item Appending song last in queue
\item Removing first element in list and returning it to the user in the process
\end{itemize}

I chose to implement the queue such that delete is never called until the destructor is called.
Every time user takes first element of queue It's immediately overwritten by the next one in queue. Because the row is replaced with it's new first entry in the first spot and last pointer moves down.
\pagebreak
\subsubsection{"Playing" songs}
The assignment told us to make the program linger a few seconds at every song before proceeding to the next. I chose the interval to be 2 seconds. I was kind of in a dilemma as there where 2 ways of doing this. I could choose between using \emph{system(sleep)} which is kind off cross-platform (i believe the windows cmd recognizes it?) But as we only use 1 thread at any given time and where told to use STL, I thought it might be cleaner (and safer) to use what I have used:
\begin{figure}[htp]
\centering
\includegraphics[scale=0.50]{/home/hergeir/Pictures/Mynd1.png}
\includegraphics[scale=0.50]{/home/hergeir/Pictures/Mynd2.png}
\caption{}
\label{}
\end{figure}

\section{Building/Compiling}
I've created a makefile for the project. It's pretty crude, but works. 
Just cd into the project directory and run make. To run the program run "make run" if you're using linux or osx. If you're using Windows your best bet is probably using visual studio.

\section{Enviroment}
I'm programming on an Arch linux 64-bit system. I've got the c++ compiler installed and compile using it's g++ alias which links necessary libraries automatically. To compile i use the recommended flags: "-std=c++11 -Wall -pedantic". The flags let me choose to use c++11 standard and give me useful compiling warnings and errors. 
For editing of code i use Gedit with syntax highlighting plugin's enabled.

\section{Backup}
And if anything's missing you can find it on: \\
github: \url{https://github.com/Hergeirs/Cpp-Obj/tree/master/Project} \\
\href{https://github.com/Hergeirs/Cpp-Obj/tree/master/Project}{Cpp-obj/Project}



\flushright{\today}
\end{document}
